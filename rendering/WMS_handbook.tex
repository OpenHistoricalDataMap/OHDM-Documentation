\section{\"Ubersicht WMS}
\begin{tabular}{ll}
\emph{Operation} & \emph{Beschreibung} \\
\hline 
GetMap & Liefert Karte(Bild) der gew\"ahlten Gegend \\ 
\hline 
Getcapabilities & Liefert Metadaten, Operationen und m\"ogliche Layer \\ 
\hline 
Exceptions & Liefert (wenn geworfen) Exceptions zur\"uck \\ 
\hline 
GetFeatureInfo(opt) & Liefert Geometrie + Attribute einer Position(px) \\ 
\hline 
DescribeLayer & Triggert \emph{WFS\footnote{Web Feature Service}} und liefert spezifischere Daten \\ 
\hline 
GetLegendGraphic & Liefert legend f\"ur die Karte \\ 
\hline 
\end{tabular} 
\section{GetMap}\begin{tabular}{lcl}
\textbf{Parameter} & \textbf{notwendig} & \textbf{Beschreibung}\\
\hline 
service & Ja & Service(WMS) \\ 
\hline 
version & Ja & Versionsnummer(1.1.0) \\ 
\hline 
request & Ja & GetMap \\ 
\hline 
layers & Ja & Anzuzeigende Ebenen \\ 
\hline 
styles & Ja & Wenn leer, wird der Standardstyle benutzt \\ 
\hline 
srs / crs & Ja & Projektionsformat(ESPG:3857) \\ 
\hline 
bbox & Ja & BoundingBox der Karte \\ 
\hline 
width & Ja & Breite der Karte in Pixel \\ 
\hline 
height & Ja & H\"ohe der Karte in Pixel \\ 
\hline 
format & Ja & Kartenformat(siehe Kartenformate)\\
\hline 
transparent & x & Transparenz der Karte(true|\emph{false}) \\ 
\hline 
bgcolor & x & Hintergrundfarbe der Karte(RRGGBB) \\ 
\hline 
exceptions & x & Exceptionformat(siehe Exceptionformate) \\ 
\hline 
time & x & Zeit(siehe Zeitformate) \\ 
\hline 
\end{tabular} \\

\textbf{Syntax}\\
/geoserver/wms?service=wms\\
request=GetMap\\
\&service=WMS\\
\&version=1.1.1\\
\&layers=ohdm\_t:boundary\_lines\\
\&styles=population\\
\&srs=EPSG:3857\\
\&bbox=-14.1,21.7,-57.15,58.96\\
\&width=900\\
\&height=500\\
\&time=yyyy-MMddThh:mm:ss.SSSZ\\
\&format=image/png\\
\hrule
\newpage
\section{GetCapabilities}
\begin{tabular}{lcl}
\textbf{Parameter} & \textbf{notwendig} & \textbf{Beschreibung}\\
\hline 
service & Ja & Service(WMS) \\ 
\hline 
version & Ja & Versionsnummer(1.1.0) \\ 
\hline 
request & Ja & GetFeatureInfo \\ 
\hline 
namespace & x &  limitiert die Suche auf den geg. Namespace\\
\hline
format & x & Format der Response\\
\hline
\end{tabular}\\

\textbf{Syntax}\\
/geoserver/wms?service=wms\\
\&version=1.1.0\\
\&request=GetCapabilities\\
\hrule
\paragraph{Requests}
des Typ \emph{GetCapabilities} kann mit Hilfe von \emph{Vendorparametern} spezifiziert werden. Die hier gelisteten Parameter sind an keine Standards und sind nur auf \emph{GeoServer} Systemen nutzbar.\\\\
\begin{tabular}{lcl}
\textbf{Vendor-Paramter} & \textbf{Beschreibung} \\
\hline 
angle  & Dreht Karte im Uhrzeigersinn um x Grad\\ 
\hline 
buffer  & Zeichnet Features, die X-Pixel ausserhalb liegen mit  \\ 
\hline 
cql\_filter &  Liste aus CQL-Filtern \\
\hline
filter & Liste aus OGC-Filtern \\
\hline

\end{tabular} 
\paragraph{Filtersyntax\\}
\textless Filter\textgreater\\
    \textless PropertyIsLessThan\textgreater\\
        \textless PropertyName\textgreater\\
            valid\_since\\
        \textless/PropertyName\textgreater\\
        \textless\ Literal\textgreater 2010-03-19Z \textless/Literal\textgreater\\
    \textless/PropertyIsLessThan\textgreater\\
\textless/Filter\textgreater\\

\paragraph{CQL-Syntax\\}
INTERSECTS(geometry, POINT (14.71 56.52))



\section{Exceptions}
\begin{tabular}{ll} 
Format & Syntax \\ 
\hline 
XML & exceptions=application/vnd.ogc.se\_xml\\ 
\hline 
INIMAGE & exceptions=application/vnd.ogc.se\_inimage \\ 
\hline 
BLANK & exceptions=application/vnd.ogc.se\_blank \\ 
\hline 
PARTIALMAP & exceptions=application/vnd.gs.wms\_partial \\ 
\hline 
JSON & exceptions=application/json \\ 
\hline 
JSONP & exceptions=text/javascript \\ 
\hline 
\end{tabular} 
\section{GetFeatureInfo}
\begin{tabular}{lcl}
\textbf{Parameter} & \textbf{notwendig} & \textbf{Beschreibung}\\
\hline 
service & Ja & Service(WMS) \\ 
\hline 
version & Ja & Versionsnummer(1.1.0) \\ 
\hline 
request & Ja & GetFeatureInfo \\ 
\hline 
layers & Ja & Anzuzeigende Ebenen \\ 
\hline 
styles & Ja & Wenn leer, wird der Standardstyle benutzt \\ 
\hline 
srs / crs & Ja & Projektionsformat(ESPG:3857) \\ 
\hline 
bbox & Ja & BoundingBox der Karte \\ 
\hline 
width & Ja & Breite der Karte in Pixel \\ 
\hline 
height & Ja & H\"ohe der Karte in Pixel \\ 
\hline 
query\_layers & Ja & Liste von zu durchsuchenden Layern(getrennt mit ,)\\
\hline 
info\_format & x & Transparenz der Karte(true|\emph{false}) \\ 
\hline 
feature\_count & x & Hintergrundfarbe der Karte(RRGGBB) \\ 
\hline 
x & x & X-Koordinate des zu untersuchenden Punktes \\ 
\hline 
y & x & Y-Koordinate des zu untersuchenden Punktes \\ 
\hline 
\end{tabular} 

\section{DescribeLayer}
\begin{tabular}{lcl} 
\textbf{Parameter} & \textbf{notwendig} & \textbf{Beschreibung} \\ 
\hline 
service & Ja & WMS \\ 
\hline 
version & Ja & 1.0.0 \\ 
\hline 
request & Ja & DescribeLayer \\ 
\hline 
Layers & Ja & Liste zu beschreibender Layer \\ 
\hline 
exceptions & x & Exceptionformat \\ 
\hline 
output\_format & x & Output Format \\ 
\hline 
\end{tabular} 
\paragraph*{Output-Formate\\}
Text : text/xml\\
GML2 :application/vnd.ogc.wms\_xml\\
JSON :application/json\\
JSONP :text/javascript\\
