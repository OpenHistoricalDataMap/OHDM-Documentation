
\documentclass[12pt]{article}
\usepackage[utf8]{inputenc}
\begin{document}
\section{Dokumentengeschichte}
\begin{table}[h]
 \begin{tabular}{|l|l|l|}
 \hline
 Zeitraum & TPL/Autor(en) & Änderungen \\
 \hline
 Wintersemester 2017/18 & Maximilian Lenz & 
  \\
 \hline
 \end{tabular}
 \caption{Dokumentengeschichte}
 \end{table}

\section{Aufgabe der Komponente}
\textbf{Web Map Services} rendern eine Karte basiert auf gewissen Parametern und liefern diese zur\"uck.
 

\section{Architektur}
\subsection{\"Uberlick}

\begin{tabular}{ll}
\emph{Operation} & \emph{Beschreibung} \\
\hline 
GetCapabilities & Liefert Metadaten, Operationen und m\"ogliche Layer \\ 
\hline 
DescribeFeatureType & Liefert Beschreibung des Features \\ 
\hline 
GetFeature & Liefert Feature(Geometrie + Attribute)\\ 
\hline 
LockFeature & Sperrt ein Feature und verhindert dessen Bearbeitung\\
\hline 
Transaction & Erstellt, Bearbeitet oder löscht ein Feature\\ 
\hline 
\end{tabular} 
\paragraph{Updates}
Ab \emph{WFS}2.0 können folgende Operationen genutzt werden:\\
\begin{tabular}{ll}
\emph{Operation} & \emph{Beschreibung} \\
\hline 
GetPropertyValue	 & Featurewert(e) abfragen mit Hilfe einer \emph{Query Expression}\\
\hline 
GetFeatureWithLock & Liefert Feature(s) und sperrt diese\\
\hline 
CreateStoredQuery	& Erzeugt eine \emph{Stored Query} auf dem Web Feature Server\\
\hline 
DropStoredQuery	& Löscht eine \emph{Stored Query} vom Web Feature Server\\
\hline 
ListStoredQueries&	Liefert eine Liste mit \emph{Stored Query} vom Web Feature Server\\
\hline 
DescribeStoredQueries	& Liefert ein Metadaten Dokument, welches die \\ & \emph{Stored Querys} beschreibt\\
\hline 
\end{tabular}
\paragraph{Deprecated} Web Feature Services der Version 1.0.0 unterstützen die \emph{GetGMLObject} Opeartion.
\\

\begin{tabular}{ll}
Operation & Beschreibung\\
\hline
GetGMLObject & Feature anhand von der \emph{ID} abfragen und als GML zurückliefern
\end{tabular}



\subsection{Schnittstellendefinitionen}
Der Wfs kann \"uber URL-Parameter und HTTP-Requests genutzt werden.\\

(Beginnen in Prototyp-Phase. Konkretisieren in der Alphaphase)

\section{Nutzung}

\subsection{GetCapabilities}
\paragraph{Grundparameter} Diese Parameter müssen bei jedem Request benötigt. Geoserver ergänzt diese werte, falls Sie nicht spezifiziert werden. Die zu ergänzenden Parameter können in der Web-Admin-Seite eingesehen werden.\\
\vspace{2em}\\
\begin{tabular}{ll}
\textbf{Parameter} & \textbf{Beschreibung}\\
\hline 
service  & Service(WFS) \\ 
\hline 
version  & Versionsnummer(1.1.0 oder 1.0.0) \\&Version darf nicht gekürzt werden (z.b. 1.0)\\ 
\hline 
request  & GetCapabilities \\ 
\hline 
\end{tabular}\\


\textbf{Syntax}\\
/geoserver/wfs?service=wfs\\
request=GetCapabilities\\
\&service=wfs\\
\&version=1.0.0\\
\&layers=ohdm\_t:boundary\_lines\\

\hrule

\paragraph{Rückgabewert}
 Der \emph{Web Feature Service} liefert eine Antwort in XML. Das Format des Dokumentes variiert mit der installierten Version.
 Dennoch folgen alle Versionen dem gleichen Grundprinzip, welches aus fünf Komponenten besteht:\\
 \vspace{2em}\\
 \begin{tabular}{ll}
 Komponente & Beschreibung\\
 \hline
 ServiceIdentification	&Enthält Headerinformationen wie etwa Title oder \\& 
 							ServiceType. Der ServiceType beschreibt welche\\& \emph{WFS} Versionen ünterstützt sind\\
\hline 							
 ServiceProvider	 & Liefert Kontaktdaten zu der Personoder Firma, \\&die den Service gepublished hat.\\&( Email, Adresse, Telefon, \ldots )\\
 \hline
 OperationsMetadata	& Beschreibt Operationen die der \emph{WFS} Server unterstützt\\& und die Parameter dieser Operationen.\\
 \hline
FeatureTypeList & Liste der \emph{FeatureTypes} des \emph{WFS} Servers. Diese werden\\& in folgender Form aufgelistet: namespace:featuretype. \\& Zudem werden Projektion Typ und Bounding Box ausgegeben\\
 \hline
 Filter\_Capabilities	 & Liste der Filter(Expressions) Beispiel: 
 						\\& SpatialOperators (Equals, Touches),
 						\\& ComparisonOperators (LessThan, GreaterThan). 
 						\\& Die Filter an sich werden nicht zurückgegeben.\\
 						\hline
 
 \end{tabular}
\subsection{DescribeFeatureType}
\begin{tabular}{lll}
\emph{Parameter} & \emph{bentötigt} & \emph{Funktion} \\
\hline
service	& Ja & 	 WFS\\
\hline
version	& Ja &	 Version (1.0.0 oder 1.1.0)\\
\hline
request	& Ja &	DescribeFeatureType\\
\hline
typeName	& Ja &	Name des Featuretypen (typeNames für WFS über 1.1.0)\\
\hline
exceptions	& x 	& Format für Exceptions(default = application)\\
\hline
outputFormat &	x	 &  Rückgabeformat (scheme description language)\\
\hline
\end{tabular}\\
\vspace{2em}\\
\textbf{Syntax}\\
/geoserver/wfs?service=wfs\\
request=DescribeFeatureType\\
\&service=wfs\\
\&version=1.0.0\\
\&typeNames=ohdm\_t:boundary\_lines\\

\hrule
\subsection{GetFeature}
\begin{tabular}{lll}
\emph{Parameter} & \emph{bentötigt} & \emph{Funktion} \\
\hline
service	& Ja & 	 WFS\\
\hline
version	& Ja &	 Version (1.0.0 oder 1.1.0)\\
\hline
request	& Ja &	GetFeature\\
\hline
typeName	& Ja &	namespace:layer (zb. ohdm\_t:boundary\_lines)\\
\hline
featureID & x & Filtert Layer nach Name\\
\hline
count & x & Limitiert die Ergebnisse (empfohlen)\\&& (count=250 ist Limit vieler Browser)\\
\hline
maxFeatures & x & Limitiert die Ergebnisse \\&&(maxFeatures=250 ist Limit vieler Browser)\\
\hline
sortBy & x & Sortiert die Features basierend auf einer Spalte \\&&der Datenbank. (zb. sortBy=valid\_since+A)\\&& +A/+D auf-/absteigend\\
\hline
propertyName & x & Liefert entsprechende Spalte der Datebank\\&&Mehrere Spalten können mit \emph{,} getrennt werden\\
\hline
srs & x & Koordinaten Referenz System (CRS)\\&& zb. srs=EPSG:3857\\
\hline
bbox & x& Boundingbox im Format a1,b1,a2,b2\\&& Darstellung abhängig vom CRS\\
\end{tabular}
\subsection{Test}
Die Komponente wird im Browser getestet.\\
Beispiel URL-Parameter:\\
\textbf{http://ohm.f4.htw-berlin.de:8080/geoserver/wfs?SERVICE=WFS\\
\&VERSION=1.1.0\\
\&REQUEST=GetFeature\\
\&typeName=ohdm\_t:ohdm\_boundary\_lines\\
\&filter=\textless Filter\textgreater \\
 \textless PropertyIsEqualTo\textgreater \\
\textless PropertyNamevalid\_since\textless /PropertyName\textgreater\\
\textless Literal\textgreater 2016-03-19Z\textless /Literal\textgreater \\
\textless /PropertyIsEqualTo\textgreater \\
\textless /Filter\textgreater\\
\&MAXFEATURES=20\\}\\

\end{document}
